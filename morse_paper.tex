\documentclass[sigplan]{acmart}

\usepackage[T1,T2A]{fontenc}
\usepackage[utf8]{inputenc}
\usepackage[greek, english]{babel}

\usepackage{booktabs} % For formal tables
\usepackage{hyperref}
\usepackage{listings}

% Copyright
%\setcopyright{none}
%\setcopyright{acmcopyright}
%\setcopyright{acmlicensed}
\setcopyright{rightsretained}
%\setcopyright{usgov}
%\setcopyright{usgovmixed}
%\setcopyright{cagov}
%\setcopyright{cagovmixed}

%Conference
\acmConference[SIGBOVIK 2020]{SIGBOVIK}{April 2020}{Pittsburgh, PA USA}
\acmYear{2020}
\copyrightyear{2020}


%\acmBadgeL[http://ctuning.org/ae/ppopp2016.html]{ae-logo}
%\acmBadgeR[http://ctuning.org/ae/ppopp2016.html]{ae-logo}

% Document starts
\begin{document}

% Title portion. Note the short title for running heads
\title[MORSE OF COURSE: Paper Reveals Time Dimension Wasted]{MORSE OF COURSE: Paper Reveals Time Dimension Wasted}

\author{Dougal Pugson}
\affiliation{%
  \institution{Pugson's C++ Crypt LLC}
  \city{New York}
  \country{USA}
}
\email{dougalpugson@gmail.com}

\begin{abstract}
All modern communication protocols have been discovered to send
nonsensical and invalid morse code sequences in addition to their
intended data. This paper demonstrates that making use of these
'wasted bits' can effectively infinitely increase the transmission
rate of binary data.
An implementation of timing-based dual-stream morse
encoding is provided in a modern programming language.
\end{abstract}

\ccsdesc[500]{Transport Protocols}
\ccsdesc[500]{Encoding}
\ccsdesc[500]{Implementation~Bash}

\keywords{SIGBOVIK, effect handler, delimited continuations, bash, shells}

\maketitle

\section{Introduction}\label{introduction}

Communication between electronic entities can be concieved of as a system
of tubes \footnote{ c.f. Stevens et alia.} or pipes \footnote{ As superbly illustrated in the inimitable
interactive video title "Super Mario Bros. 2".}.
Into these pipes, electronic satchels\footnote{So-called ``packets''.} are inserted
by the electronic system. The contents of these ``packets'' are composed
according to a specified algorithmic protocol\footnote{ From the Byzantine \begin{otherlanguage*}{greek}πρωτόκολλον\end{otherlanguage*}, meaning "First Page",
referring to the average amount of the design specification Engineers are expected to read
before beginning implementation.}, in a way such that, when recieved by the
recipient \footnote{recipiō, recipere, recēpī, receptum.} can be reassembled into
the desired message\footnote{%
As expressed by the immortal MARSHALL McCLUHAN in the ground-breaking epistle
"The Media is the Message",

\begin{quote}
In a culture like ours, long accustomed to splitting and 
dividing all things as a means of control, it is sometimes
a bit of a shock to be reminded that, in opera- tional
and practical fact, the medium is the message.
\end{quote}}.

Many protocols for sending data via the computer have been developed,
such as TCP, FTP, HTTP, et cetera. Upon inspection of these protocols,
it was discovered that, in addition to their intentional messages,
they were also transmitting nonsensical and invalid morse code
suquences.

Let ``.'' represent a short transmission, and ``-'' represent a long transmission.
An example transmission measured from a typical TCP communication is as follows:

\begin{quote}
.......................................................
\end{quote}.

This invalid morse code sequence is unparseable\footnote{``sssssssssssssssssss'' is one possible interpretation}.

This paper will demonstrate that this wasted information can be
meaningfully replaced with useful data, effectively doubling
the effective transmission rate.

\subsection{Mathematical Prolegomena}

Let $\mathfrak{D}$ represent the number of bits per transmission
in the ordinary dimension (1 vs 0).

Let $\mathfrak{M}$ represent the number of usable bits per transmission
in the temporal dimension.

Let $A_\alpha$ represent the average packet transmission length.

Let $c$ represent the fastest average length of time between each
transmission packet sent via the transmission tube (that is, the limit provided
by the combination of hardware construction and the speed of light
in an average vacuum)\footnote{ Though, as demonstrated by Dyson et al, brand and model can significantly
affect results.}.

Then, the rate of data transfer $r$ of any protocol may be expressed as follows:

\begin{eqnarray*}
    r = \frac{\mathfrak{M}}{A_\alpha} + \frac{\mathfrak{D}}{A_\alpha} \\
\end{eqnarray*}

For typical transmissions protocols, these following values hold:
$\mathfrak{M} = 0$, $\mathfrak{D} = 1$, and $A_\alpha = c$
With these values known, the equation may then be evaluated.

\begin{eqnarray*}
    r = \frac{\mathfrak{M}}{A_\alpha} + \frac{\mathfrak{D}}{A_\alpha} \\
    r = \frac{0}{c} + \frac{1}{c} \\
    r = \frac{1}{c} \\
\end{eqnarray*}

With the transmission rate of standard protocols established,
we now examine the transmission rate of a temporal encoding.
In order for temporal encoding, i.e., morse code, to be successfully transmitted,
pauses will be required to be inserted into the transmission stream.
This means that, for such encodings,

\[
    \mathfrak{M} > 1 \models A_\alpha < c
\]

Let us assume that each transmission unit contains one bit of information,
and that, additionally, that transmission unit may be either short (a delay
of 0) or long (a delay of some abritrary value $\lambda$). This would
make the information value $\mathfrak{M}$, as defined above, 2.
Given this, we may again evaluate the transmission rate equation
with the following values in order to calculate the transmission
rate of our new temporal encoding:
$\mathfrak{M} = 2$, $\mathfrak{D} = 1$, and $A_\alpha = c + \lambda$

\begin{eqnarray*}
    r = \frac{\mathfrak{M}}{A_\alpha} + \frac{\mathfrak{D}}{A_\alpha} \\
    r = \frac{1}{c + \lambda} + \frac{1}{c + \lambda} \\
    r = \frac{2}{c + \lambda} \\
\end{eqnarray*}

With a small value $\lambda$, we may drastically increase throughput, up to
an effective doubling.

As time is continuous,
any given time interval may be divided into infinitely many fine
gradations, e.g., long, short, very short, very very short ... $very_\infty$ short.

Thus, $\mathfrak{M}$ may be arbitrarily large. The industrial
applications of this surprising fact should not be lost upon
the reader.

\subsection{Implementation}

The author has provided an model implementation
of timing-based encoding in the modern programming
bash. The full source code of this program may be
found at 

\begin{lstlisting}[language=bash]
function usage {
    echo "MORSE CODE SENDER"
    echo "morse.sh --send    input1.txt  input2.txt"
    echo "morse.sh --recieve output1.txt output2.txt"
}
\end{lstlisting}

\subsection{Conclusion conclusion}

\bibliographystyle{ACM-Reference-Format}
\bibliography{bibliography}

\end{document}
